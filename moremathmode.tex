\chapter{More Math Mode}
\label{cha:more-math-mode}

\section{Matrices}
\label{sec:matrices}

\LaTeX{} provides several different environments for making matrices
and table-like structures.  Common to all of these environments is the
use of the ampersand, \&, to delineate columns and the line break
\verb_\\_ to separate rows.  Each row must be on its own line, though
columns need not line up.  Although it is not necessary to maintain
any manner of alignment, maintaining any manner of alignment of the
columsn between rows can be helpful, especially in larger tables.

Let us start by making a simple matrix using the \verb_array_
environment.  Note that to use the \verb_array_ environment, one must
be in math mode first.  Suppose we want to type the following matrix:
\[
\left(\begin{array}{rr}
    1 & 0 \\
    0 & 1
  \end{array} \right)
\]
Then we want an array with a 1 and a 0 in the first row, and a 0 and a
1 in the second row.  Thus, the code is
\begin{verbatim}
\begin{array}{rr}
  1 & 0 \\
  0 & 1
\end{array}
\end{verbatim}
However, this will not enclose the numbers in parentheses.  As
previously noted, the large left and right parentheses can be typeset
using the commands \verb_\left_ and \verb_\right_.  Also, note that
\verb_{rr}_ has been appended to \verb_\begin{array}_.  This indicates
the vertical alignment of the columns.  The first \texttt{r}
indicates the first column should be right-aligned, while the second
\texttt{r} indicates the same for the second column.  Text in the
columns can be aligned left, right, or center using \texttt{l},
\texttt{r}, and \texttt{c}, respectively.  Thus
\begin{verbatim}
\left(
\begin{array}{rr}
  1 & 0 \\
  0 & 1
\end{array}
\right)
\end{verbatim}
gives the completed matrix.

The \AmS{} packages provide various environments for matrices beyond
\verb_array_.  \verb_matrix_ is essentially the same as \verb_array_,
as it creates a matrix without delimiters.  However, \verb_matrix_
automatically centers the entries within the columns.  \verb_pmatrix_
and \verb_vmatrix_ are two other matrix environments, and they come
with delimiters.  Consider our previous example matrix written using
\verb_pmatrix_ and \verb_vmatrix_, respectively:
\[
\begin{pmatrix}
  1 & 0 \\
  0 & 1
\end{pmatrix}
\begin{vmatrix}
  1 & 0 \\
  0 & 1
\end{vmatrix}
\]
Additionally, \verb_matrix_ and similar environments can only handle
matrices of up to ten columns.  Should you need a matrix with more
columns, precede the start of the matrix environment with
\begin{verbatim}
\setcounter{MaxMatrixCols}{numcolumns}.
\end{verbatim}

\section{Commutative Diagrams}
\label{sec:commutative-diagrams}

While extensive graphics work in \LaTeX{} becomes something of a
computer geek black art, commutative diagrams can be drawn easily
using the \verb_\xymatrix{}_ command.  This command is part of the
\verb_xy_ package by putting
\begin{verbatim}
\usepackage[all]{xy}
\end{verbatim}
in the preamble.  The \verb_\xymatrix{}_ command is used inside math
mode, and behaves much like the \verb_array_ environment.  For
example,
\begin{verbatim}
\[
\xymatrix{G & H \\
J & K}
\]
\end{verbatim}
will give
\[
\xymatrix{G & H \\
  J & K}
\]
These can all be connected with arrows using the \verb_\ar[argument]_
command.  The argument of the command indicates the direction of the
arrow: \texttt{l}, \texttt{r}, \texttt{u}, and \texttt{d} for left,
right, up, and down, respectively.  These directions can be combined,
repeatedly if necessary.  For example, \verb_\ar[drrr]_ points down
three columns to the right.  The source code for a basic commutative
diagram would look like this:
\begin{verbatim}
\[
\xymatrix{G \ar[r] \ar[d] & H \ar[d] \\
J \ar[r] & K}
\]
\end{verbatim}
  This gives the diagram
  \[
  \xymatrix{G \ar[r] \ar[d] & H \ar[d] \\
    J \ar[r] & K}
  \]
  Furthermore, these arrows can be labeled using subscripts and
  superscripts.  For vertical arrows, subscripts go to the left, and
  superscripts go to the right.
\begin{verbatim}
\[
\xymatrix{G \ar[r]_f \ar[d]^\tau & H \ar[d]_\sigma \\
J \ar[r]_g & K}
\]
\end{verbatim}
\[
\xymatrix{G \ar[r]_f \ar[d]^\tau & H \ar[d]_\sigma \\
J \ar[r]_g & K}
\]

\section{Piecewise Functions}
\label{sec:piecewise-functions}

The typesetting of piecewise functions is accomplished 
by using a built-in environment from the \verb_amsmath_
environment.  Piecewise functions should be typeset almost exclusively
in displayed math environments.  It is possible to place a piecewise
function in an inline math environment, but the large size of a
piecewise function will make a mess of your page.

The \verb_amsmath_ package includes a \verb_cases_
environment for typesetting piecewise functions and other related
functions. It works just like the \verb_matrix_ environment (also
provided by \verb_amsmath_), but only two columns are permitted.
The left brace is also automatically printed before the array.
Here is an example:
\begin{verbatim}
\[
\delta(x) = \begin{cases}
  0 & \mbox{if } x \neq 0 \\
  \infty & \mbox{if } x = 0
\end{cases}
\]
\end{verbatim}
\[
\delta(x) = \begin{cases}
  0 & \mbox{if } x \neq 0 \\
  \infty & \mbox{if } x = 0
\end{cases}
\]

\section{Better Looking Math}
\label{sec:better-looking-math}

There are several commands that are very useful for improving the
appearance of your mathematical content.

\subsection{Eqnarray}
\label{sec:eqnarray}

\emph{Warning: This environment is deprecated and has
  been supplanted by the \texttt{align} environment provided by
  \texttt{amsmath}. It is only mentioned here for historical reasons.}

The \texttt{eqnarray} environment is similar to the \texttt{array} or
\texttt{tabular} environment, but is at most three columns wide, and
no alignment needs to be declared for the columns.  Also,
\texttt{eqnarray} is a math environment, meaning no dollar signs or
other math mode environments are necessary.  The environment comes in
two flavors: \texttt{eqnarray} will number the rows, while
\texttt{eqnarray*} will not.  The environment is entered and exited
like any other with \verb|\begin{eqnarray*}| and
\verb|\end{eqnarray*}|.  Here is an example:
\begin{verbatim}
\begin{eqnarray}
  \label{eq:2}
  3x+2&=&5 \\
  3x&=&3 \\
  x&=&1
\end{eqnarray}
\end{verbatim}
which results in:
\begin{eqnarray}
  \label{eq:2}
  3x+2&=&5 \\
  3x&=&3 \\
  x&=&1
\end{eqnarray}

\subsection{Align}
\label{sec:align}

\emph{Note: This environment is the recommended way to typeset
  aligned equations.}

The \texttt{align} environment, like \texttt{eqnarray}, is similar to
the \texttt{array} or \texttt{array} environment, but unlike the
\texttt{eqnarray} environment, this environment is provided by the
\texttt{amsmath} package and offers more features than the provided
\texttt{eqnarray} environment.  Here is an example of the basic syntax
of the \texttt{align} environment:
\begin{verbatim}
\begin{align}
  \label{eq:3}
  3x+2&=5 \\
  3x&=3 \\
  x&=1
\end{align}
\end{verbatim}
which ends up as
\begin{align}
  \label{eq:3}
  3x+2&=5 \\
  3x&=3 \\
  x&=1
\end{align}

\subsection{Displaystyle}
\label{sec:displaystyle}

The \verb|\displaystyle| command is the best friend of inline math.
For the most part, \verb|\displaystyle| will make things bigger.  It
will make inline fractions more readable, and for inline integrals,
and general sums and products, it will place the indices above and
below, rather than to the side of the operator.  The use of the
\verb|\displaystyle| is best described with a few examples.  In the
following example, the compiled code will immediately follow the source.
\begin{verbatim}
Some inline math: $\frac{\sqrt{2}}{\sqrt{3}}$.
\end{verbatim}

Some inline math: $\frac{\sqrt{2}}{\sqrt{3}}$.

\begin{verbatim}
Now with \texttt{displaystyle}:
$\displaystyle\frac{\sqrt{2}}{\sqrt{3}}$.
\end{verbatim}

Now with \texttt{displaystyle}: $\displaystyle\frac{\sqrt{2}}{\sqrt{3}}$.

\begin{verbatim}
A bad looking sum: $\sum_{n=1}^\infty \frac{1}{n}$.
\end{verbatim}

A bad looking sum: $\sum_{n=1}^\infty \frac{1}{n}$.

\begin{verbatim}
\texttt{displaystyle} to the rescue!
$\displaystyle \sum_{n=1}^\infty \frac{1}{n}$.
\end{verbatim}

\texttt{displaystyle} to the rescue! $\displaystyle \sum_{n=1}^\infty
\frac{1}{n}$.

Note that \verb|\displaystyle| need only be used once in each math
environment.

\section{Theorems and Proofs}
\label{sec:theorems-proofs}

\subsection{Theorems}
\label{sec:theorems}

Back in the dark ages of \TeX{}\footnote{One should not advertise that
one actually used \TeX{}, especially if \LaTeX{} was written before
one was born, as one will get strange looks.}, the command to begin a
theorem was called \texttt{proclaim}.  However, \LaTeX{} isn't as
overjoyed about its theorems, so \texttt{proclaim} is gone and
replaced by the \texttt{theorem} environment.  If you open a math
book, you'll notice that the theorems and theorem-like objects are all
labeled differently.  Some are called theorems, others lemmas and
still others propositions, among other things.  In the preamble of
your document, you need to set up each of these different names
separately using the \verb|\newtheorem| command.

First, let's look at a theorem that just has a number, rather than a
specific name:
\begin{theorem}
  Let $V$ be a finite-dimensional vector space, and define
  $\psi:V\rightarrow V^{**}$ by $\psi(x)=\hat{x}$. Then $\psi$ is an
  isomorphism.
\end{theorem}

The first thing we had to do to typeset this fine theorem was define
an environment for it via \texttt{newtheorem}.  The command has two
arguments, taking the form
\verb|\newtheorem|\{\textit{name}\}\{\textit{Name}\}.  The first argument
names the environment, and the second argument determines what things
will be labeled.  (The two arguments do not have to be the same, but
it would be a bit weird for the \texttt{chicken} environment to be
producing theorems.)  In our example, the full command looked like
\verb|\newtheorem{theorem}{Theorem}|.  The remainder of the code was
\begin{verbatim}
\begin{theorem}
  Let $V$ be a finite-dimensional vector space, and
  define $\psi:V\rightarrow V^{**}$ by $\psi(x)= %
  \hat{x}$. Then $\psi$ is an isomorphism.
\end{theorem}
\end{verbatim}
But what happens if your theorem has a name?
\begin{theorem}[Fermat's Last Theorem]
  If an integer $n$ is greater than 2, then the equation $a^n + b^n =
  c^n$ has no solutions in non-zero integers $a$, $b$, and $c$.
\end{theorem}

The only difference between this and the previous example is the
addition of the name.  We didn't have to add another
\verb|\newtheorem| because we had already defined a theorem
environment, which is also why this is numbered as Theorem 2.  So how
did we add the name?  When we opened the \texttt{theorem} environment,
we did it with
\begin{verbatim}
\begin{theorem}[Fermat's Last Theorem]
\end{verbatim}
This is all well and good, but what if you have a theorem that is so
fantastically cool giving it a number would be cheapening the moment?
\begin{morse}
  Let $p_0$ be a non-degenerate critical point of a function $f$ of
  two variables.  Then we can choose appropriate local coordinates
  $(X,Y)$ in such a way that the function $f$ expressed with respect
  to $(X,Y)$ takes one of the following three standard forms:
  \begin{enumerate}
  \item $f=X^2+Y^2+c$
  \item $f=X^2-Y^2+c$
  \item $f=-X^2-Y^2+c$
  \end{enumerate}
\end{morse}

If you recall, the way we got rid of numbering on other things was to
add a *\@.  The question is, do you add it to \verb|\newtheorem| or
the start of the environment?  From experience, it seems as if adding
it to \verb|\newtheorem| would ensure that every instance of that
environment will be unnumbered, and adding it to the start of the
environment would just eliminate the number from that one instance.
As it happens, the answer is \verb|\newtheorem|, sort of.  If you just
tack on the asterisk, you'll get an error as \verb|\newtheorem*| is
not defined in \LaTeX{}.  However, it does exist with the inclusion of
the \texttt{amsthm} package.  To typeset the Morse Lemma, we declared
the environment with \verb|\newtheorem*{morse}{Morse Lemma}|.

Each different proclamation environment has its own counter by
default, so you could have a Theorem 1, a Definition 1, a Lemma 1, and
who knows what else numbered 1.  But what if you wanted your theorems
and lemmas numbered in succession so Theorem 1 was followed by Lemma 2
rather than Lemma 1?  Covering the entire scope of counters would be
unmanageable at this juncture, so we'll jsut cover the very basics and
leave the rest for another time and place (quite possibly another
book).  As it turns out, achieving this end leads us back to the
\verb|\newtheorem| command.  So how do you declare the lemma
environment?
\begin{verbatim}
\newtheorem{lemma}[theorem]{Lemma}
\end{verbatim}
There's one potential pitfall with this system.  What happens if
\texttt{theorem} has not been defined yet?  Bad, bad things happen, so
make sure you tie the counting to something that already exists.

There is another counting option that ties numbering to some
subdivision of the document, creating Theorem 1.1, etc.  This option
is of the form
\begin{verbatim}
\newtheorem{theorem}{Theorem}[section]
\end{verbatim}
where \emph{section} can be any subdivision of the document such as
\texttt{section}, \texttt{chapter}, or \texttt{subsection}.  Using
this numbering scheme and tying the numbering to the subsection, we
can create:
\begin{thm}
  Every bounded, monotonic sequence is convergent.
\end{thm}

The \texttt{amsthm} package provides the
\verb|\newtheoremstyle|\{\textit{style}\} command which lets you
define what the theorem is all about by picking from three predefined
styles: \texttt{definition} (fat title, roman body), \texttt{plain}
(fat title, italic body), or \texttt{remark} (italic title, roman
body).\footnote{Oetiker, Tobias, Hubert Partl, Irene Hyna, and
  Elisabeth Schlegl.  \emph{The Not So Short Introduction to
    \LaTeXe{}}.  May 2006.  p.\ 60}

\subsection{The Proof Environment}
\label{sec:proof-environment}

In addition to allowing for unnumbered theorems, the \texttt{amsthm}
package provides the \texttt{proof} environment.  Using the
environment is fairly straightforward; just follow the
\verb|\begin{proof}| with what you want the proof labeled in square
  brackets.  For example, if youw ant to include the proof of the
  Morse Lemma, your code would look something like:
\begin{verbatim}
\begin{proof}[Proof of Morse Lemma]
  the proof goes here
\end{proof}
\end{verbatim}
The \texttt{proof} environment automatically places the QED symbol
($\Box$) at the end, leaving an appropriate amount of space in
front.  However, should it place the symbol in the wrong place, you
can force the placement with \verb|\qedhere|.

%%% Local Variables: 
%%% mode: latex
%%% TeX-master: "reader"
%%% End: 
