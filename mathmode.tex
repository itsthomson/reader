\chapter{Math Mode Basics}
\label{cha:math-mode-basics}

One common usage of \LaTeX{} is the typesetting of mathematical
discourse and documents with mathematical content.  \LaTeX{} comes
with a good deal of functionality in this area and has become a fairly
standard tool in the math community.  In addition, the math
typesetting abilities of \LaTeX{} can be further expanded by the use
of the \AmS{-}\LaTeX{} packages.

Packages in general will be discussed later, but for now, it suffices
to add
\begin{verbatim}
\usepackage{amsmath,amsfonts,amssymb,amsthm}
\end{verbatim}
to the preamble of any document you are typesetting with math content.
This places additional environments and symbols at your disposal that
make life far easier.

\section{Math Mode}
\label{sec:math-mode}

The key tool for typesetting mathematical content is math mode.  Math
mode is accessed through a variety of math environments.  In math
mode, the basic behavior of \LaTeX{} is altered in a few important
ways.  First of all, the letters on the keyboard become constants and
variables.  For example, an \verb=a= in math mode will be displayed as
$a$ while in normal text, it is an a.  This gives them an italicized
appearance to make them typographically different from their textual
counterparts.  Note that this is not the same as typesetting text in
italics, and math mode should never be used to italicize text.

While in math mode, the fast majority of characters on your keyboard
will continue to function as normal.  In addition to the letters
available on the keyboard, there are a wide variety of other
characters and symbols available through various \LaTeX{} commands.
\LaTeX{} also offers an extensive array of hats, bars, and other
useful accents.

The math mode chapters do not have many examples.  However, there is a
selection of examples following the chapters.

\section{Math Environments}
\label{sec:math-environments}

\LaTeX{} is extremely powerful, not least in its ability to typeset
math.  There are three math environments: \texttt{math},
\texttt{displaymath}, and \texttt{equation}.  The inline environment
allows the inclusion of math in a normal line of text, while the
displayed environment enables one to set off lines of math.  The
equation environment is essentially the same as the displayed
environment, but each instance is numbered sequentially.  $x+y=z$ is
an example of the inline environment, while \[x+y=z\] is the same
thing in the displayed environment.

\subsection{Inline Math}
\label{sec:inline-math}

Like other \LaTeX{} environments, inline math, or more accurately, the
\texttt{math} environment, is delimited by \verb=\begin= and
\verb=\end= commands.  In \TeX{}, inline math was set off by dollar
signs.  This convention was carried over to \LaTeX{}, which allows us
a shortcut, as all those \verb=\begin= and \verb=\end=s would get very
tedious very quickly.  Using the \$ convention, the above example
would appear in one's source as
\begin{verbatim}
$x+y=z$ is an example of the inline environment.
\end{verbatim}

\subsection{Displayed Math}
\label{sec:displayed-math}

The displayed math environment is formally delimited by
\begin{verbatim}
\begin{displaymath}
\end{displaymath}
\end{verbatim}
However, one can also use $\backslash [$ at the beginning and
$\backslash ]$ at the end of the statement.  Thus, using square
brackets as the delimiters, \[x+y=z\] would appear in one's source as
\begin{verbatim}
\[x+y=z\]
\end{verbatim}

\subsection{Equations}
\label{sec:equations}

The equation environment returns to the conventional delimiters for
environments.  It is accessed by \verb=\begin{equation}= and
\verb=\end{equation}=.  Our example equation in the equation
environment would appear as
\begin{equation}
  \label{eq:1}
  x+y=z
\end{equation}

\subsection{Stars (Well, Asterisks Really)}
\label{sec:stars-well-asterisks}

Various environments and structures in \LaTeX{} will automatically
number themselves, like the \texttt{equation} environment above.
However, unlike \texttt{equation}, not everything has a separate
unnumbered counterpart.  To avoid the numbering, an asterisk (*) is
used to disable numbering.  This fact is of little concern now, but it
is a fact to file away, as it will come up again later.

\section{Commands}
\label{sec:commands}

If you take a look at the keyboard, you'll see a number of symbols
that look fairly useful for typing math, including +, -, *, /, =, (,
and ), essentially the same things you find on a small calculator.
Which is all well and good, up to a point, but what about typing
homework?  Or textbooks for that matter, as your math textbooks were
most likely done with \LaTeX{}.  It is possible to include symbols in
math environments that are not available on one's keyboard.  This is
achieved through various commands.  With two important exceptions,
math commands, like many other \LaTeX{} commands, begin with a
backslash, $\backslash$.

\subsection{Subscripts and Superscripts}
\label{sec:subscr-superscr}

The two exceptions to the commands beginning with a backslash are \^{}
and \_, the commands for superscript and subscript, respectively.
Thus, $x^2$ is entered as \verb=x^2= and $x_2$ as \verb=x_2=.  Note
that \^{} and \_ only include the character immediately following,
unless one encloses the subscript or superscript in brackets.
Therefore, $x^{12}$ is entered as \verb=x^{12}=, not as \verb=x^12=,
which is $x^12$.

\subsection{Math Symbols}
\label{sec:math-symbols}

If you think back to the last homework assignment you did or the last
textbook you read, you'll probably remember that math uses all sorts
of symbols such as $\infty$, $\sum$, $\int$, and a whole host of
non-Roman letters.  Appendix~\ref{cha:commonly-used-math} contains a
table of various math symbols.  The commands work in essentially the
same way as other \LaTeX{} commands.  Just remember that they only
work in math mode.

Frequently, it seems like a lot of bother to go sifting through
appendices looking for a command.  Fortunately, many of the commands
have fairly intuitive names, so if you're willing to risk errors, you
can be lazy and guess.  At this point, you might be thinking, ``You
said think about my last homework.  I didn't use half that stuff.  I
had Greek letters, hats, and transposes coming out of my ears!''  We
will get there in a few short paragraphs, so don't worry.

\subsection{Delimiters}
\label{sec:delimiters}

In a very basic sense, delimiters are things like parentheses and
brackets.  Parentheses and square brackets can be typeset with the
matching key on the keyboard.  Other delimiters have specific
commands.  Beyond the commands for different delimiters, it is
important to know how to control the size of delimiters.  Let's say
you wanted to put parentheses around a summation for some
reason.  \[(\sum_{k=0}^{10} k)\] looks pretty silly because the
parentheses are so small.  We can use \verb=\left= and \verb=right= to
correct this problem.  When immediately followed by the approximate
delimiter (i.e., one uses ( with \verb=\left=), \LaTeX{} will size
the delimiter to what it thinks is the right size.  By writing
\begin{verbatim}
\left(\sum_{k=0}^{10}k\right)
\end{verbatim}
we get \[\left(\sum_{k=0}^{10}k\right)\] which looks a lot better.
It is important to note that the left and right hand delimiters
must be paired.  For example, attempting to compile
\verb?$\left($? will result in an error.  It is possible to have a
single left or right delimiter by pairing it with a
\verb=\left.= or \verb=\right.= as appropriate.  If you only
want a delimiter one one side, you can use \verb=\left= or
\verb=\right= without a delimiter to complete the pair.  You
also have the option of choosing the size of a delimiter with
\verb=\big=, \verb=\Big=, \verb=\bigg=, and \verb=\Bigg=, which
are used like \verb=\left= and \verb=\right= and give
progressively larger delimiters.  The existence of these
commands may seem a little puzzling, given that \LaTeX{} will
size delimiters on its own, if you ask it to.  However, there
are times when you will want to do it yourself.
\begin{table}
  \centering
  \caption{Delimiters (those with commands)}
  \label{tab:delimiters}
  \begin{tabular}{@{}ll@{}}
    \toprule
    Delimiter & Command \\
    \midrule
    \{ & \verb=\{= \\
    \} & \verb=\}= \\
    $\backslash$ & \verb=\backslash= \\
    $\langle$ & \verb=\langle= \\
    $\rangle$ & \verb=\rangle= \\
    $\|$ & \verb?\Vert? or \verb?\|? \\
    $\lfloor$ & \verb?\lfloor? \\
    $\rfloor$ & \verb?\rfloor? \\
    $\lceil$ & \verb?\lceil? \\
    $\rceil$ & \verb?\rceil? \\
    \bottomrule
  \end{tabular}
\end{table}

\section{More Math Commands}
\label{sec:more-math-commands}

Some of the major commands that take arguments are \verb?\frac? and
\verb?\sqrt?, which typeset fractions and roots, respectively.  One
writes a root by \verb?\sqrt{expr}?, where \texttt{expr} is replaced
by what goes under the root.  If one wants the $n$th root, the full
command is actually \verb?\sqrt[n]{expr}?, so $\sqrt[3]{x}$ is written
as \verb?\sqrt[3]{x}?.  Likewise, fractions take the form of
\verb?\frac?\{\textit{num}\}\{\textit{denom}\}.  Combinations, such as
$\binom{a}{b}$ are written as
\{\textit{combinations}\}\verb?\choose?\{\textit{objects}\} or
\verb?\binom{combo}{obj}?.

\subsection{Log-like Operators}
\label{sec:log-like-operators}

Occasionally, you will want to type things like \texttt{sin} and
\texttt{det}, things that are both words and symbols.  \LaTeX{}
provides commands for such operators to preclude any need to exit math
mode and to allow for appealing spacing (shown in
Table~\ref{tab:log-operator}).
\begin{table}
  \centering
  \caption{Log-like operators}
  \label{tab:log-operator}
  \begin{tabular}{@{}llll@{}}
    \toprule
    Operator & Command & Operator & Command \\
    \midrule
    $\sin$ & \verb?\sin? & $\exp$ & \verb?\exp? \\
    $\cos$ & \verb?\cos? & $\gcd$ & \verb?\gcd? \\
    $\tan$ & \verb?\tan? & $\hom$ & \verb?\hom? \\
    $\csc$ & \verb?\csc? & $\inf$ & \verb?\inf? \\
    $\sec$ & \verb?\sec? & $\ker$ & \verb?\ker? \\
    $\cot$ & \verb?\cot? & $\lg$ & \verb?\lg? \\
    $\cosh$ & \verb?\cosh? & $\lim$ & \verb?\lim? \\
    $\sinh$ & \verb?\sinh? & $\liminf$ & \verb?\liminf? \\
    $\tanh$ & \verb?\tanh? & $\limsup$ & \verb?\limsup? \\
    $\coth$ & \verb?\coth? & $\ln$ & \verb?\ln? \\
    $\arcsin$ & \verb?\arcsin? & $\log$ & \verb?\log? \\
    $\arccos$ & \verb?\arccos? & $\max$ & \verb?\max? \\
    $\arctan$ & \verb?\arctan? & $\min$ & \verb?\min? \\
    $\arg$ & \verb?\arg? & $Pr$ & \verb?\Pr? \\
    $\deg$ & \verb?\deg? & $\sup$ & \verb?\sup? \\
    $\det$ & \verb?\det? & $\dim$ & \verb?\dim? \\
    \bottomrule
  \end{tabular}
\end{table}

\subsection{Accents}
\label{sec:accents}

Another group of math commands are what one might describe as
modifiers for letters.  This would include bars and hats.  It might
not come as a surprise that the command for $\hat{x}$ is
\verb?\hat{x}?.  There is also \verb?\widehat?, which looks a bit
better on capital letters.  Compare $\hat{T}$ to $\widehat{T}$.  This
happens because \verb?\widehat? extends over everything enclosed in
the brackets, whereas \verb?\hat?  simply centers a hat over what's
enclosed in the brackets.  As a result, one can have $\widehat{xyz}$
(\verb?\widehat{xyz}?)  rather than $\hat{xyz}$ (\verb?\hat{xyz}?).
\verb?\tilde?  works the same way as \verb?\hat?.  There is also
\verb?\widetilde?.  $\bar{x}$ is obtained by \verb?\bar{x}?.  It is
worth noting that \verb?\widebar? does not exist.  Use
\verb?\overline? to accomplish a similar effect.

\section{Greek and Other Fancy Letters}
\label{sec:greek-other-fancy}

\subsection{Greek Letters}
\label{sec:greek-letters}

For the Greek alphabet, the commands are simply a backslash followed
by the name of the letter.  For example, $\beta$ is \verb?\beta?.  As
always, commands are case sensitive, so \verb?\Gamma? is $\Gamma$
rather than $\gamma$.
\begin{table}
  \centering
  \caption{Uppercase Greek letters}
  \label{tab:uppercase-greek}
  \begin{tabular}{@{}llll@{}}
    \toprule
    Letter & Command & Letter & Command \\
    \midrule
    $\Gamma$ & \verb?\Gamma? & $\Sigma$ & \verb?\Sigma? \\
    $\Delta$ & \verb?\Delta? & $\Upsilon$ & \verb?\Upsilon?
    \\
    $\Theta$ & \verb?\Theta? & $\Phi$ & \verb?\Phi? \\
    $\Lambda$ & \verb?\Lambda? & $\Psi$ & \verb?\Psi? \\
    $\Xi$ & \verb?\Xi? & $\Omega$ & \verb?\Omega? \\
    $\Pi$ & \verb?\Pi? & & \\
    \bottomrule
  \end{tabular}
\end{table}
\begin{table}
  \centering
  \caption{Lowercase Greek letters}
  \label{tab:lowercase-greek}
  \begin{tabular}{@{}llll@{}}
    \toprule
    Letter & Command & Letter & Command \\
    \midrule
    $\alpha$ & \verb?\alpha? & $\nu$ & \verb?\nu? \\
    $\beta$ & \verb?\beta? & $\xi$ & \verb?\xi? \\
    $\gamma$ & \verb?\gamma? & $\pi$ & \verb?\pi? \\
    $\delta$ & \verb?\delta? & $\rho$ & \verb?\rho? \\
    $\epsilon$ & \verb?\epsilon? & $\sigma$ & \verb?\sigma?
    \\
    $\zeta$ & \verb?\zeta? & $\tau$ & \verb?\tau? \\
    $\eta$ & \verb?\eta? & $\upsilon$ & \verb?\upsilon? \\
    $\theta$ & \verb?\theta? & $\phi$ & \verb?\phi? \\
    $\iota$ & \verb?\iota? & $\chi$ & \verb?\chi? \\
    $\kappa$ & \verb?\kappa? & $\psi$ & \verb?\psi? \\
    $\lambda$ & \verb?\lambda? & $\omega$ & \verb?\omega? \\
    $\mu$ & \verb?\mu? & & \\
    \bottomrule
  \end{tabular}
\end{table}

\subsection{Fancy Fonts}
\label{sec:fancy-fonts}

The \AmS{} packages provide several fonts that allow for some rather
useful letters and symnbols that are not available through previously
discussed commands.  These fonts are selected by
\verb?\[font command]{text}?.  This is most easily illustrated by an
example such as \verb?$\mathbb{R}$?.  This command typesets a capital
R in the math board bold font.  Compiled, it gives $\mathbb{R}$, the
familiar R for the reals.

Other fonts available from \LaTeX{} and the \AmS{} packages are shown
in Table~\ref{tab:math-fonts}.
\begin{table}
  \centering
  \caption{Math fonts}
  \label{tab:math-fonts}
  \begin{tabular}{@{}ll@{}}
    \toprule
    Font & Command \\
    \midrule
    $\mathbb{NQRZ}$ & \verb|\mathbb{}| \\
    $\textrm{text}$ & \verb|\textrm{}| \\
    $\text{text}$ & \verb|\text{}| \\
    $\mathrm{math}$ & \verb|\mathrm{}| \\
    $\mathbf{bold}$ & \verb|\mathbf{}| \\
    $\mathsf{ABCXYZ}$ & \verb|\mathsf{}| \\
    $\mathit{italics}$ & \verb|\mathit{}| \\
    $\mathcal{ABCXYZ}$ & \verb|\mathcal{}| \\
    $\mathfrak{ABCXYZ}$ & \verb|\mathfrak{}| \\
    \bottomrule
  \end{tabular}  
\end{table}

The board bold font is the most commonly used one.  Also of note are
\verb?\textrm{}? and \verb?\mathrm{}?.  Both of these fonts override
the default italicization in math mode.  \verb?\textrm{}? typesets
text in a style identical to normal text mode and \verb?\mathrm?
typesets text in normal upright letters, but slightly different for
math.  A more detailed discussion of these differences can be found in
Oetiker's \emph{The Not So Short Introduction to \LaTeX{}}.  Refer to
Appendix~\ref{cha:commonly-used-math} for examples of some of these
fonts.  If you would like to check on any of the other fonts, just
experiment with a \LaTeX{} compiler.  Be aware that some of the fonts
only work for capital letters.

\subsection{Dots}
\label{sec:dots}

An ellipsis (three dots) is often handy in math mode.  \LaTeX{}
provides a wide variety of ellipses.  They are shown in
Table~\ref{tab:dots}.
\begin{table}
  \centering
  \caption{Dots}
  \label{tab:dots}
  \begin{tabular}{@{}ll@{}}
    \toprule
    Symbol & Command \\
    \midrule
    \ldots & \verb?\ldots? \\
    $\cdots$ & \verb?\cdots? \\
    $\vdots$ & \verb?\vdots? \\
    $\ddots$ & \verb?\ddots? \\
    \bottomrule
  \end{tabular}
\end{table}

%%% Local Variables: 
%%% mode: latex
%%% TeX-master: "reader"
%%% End: 