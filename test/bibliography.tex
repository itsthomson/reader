\chapter[Bibliographies]{Bibliographies for Fun and Profit}
\label{cha:bibl-fun-prof}

\section{Overview}
\label{sec:overview}

\LaTeX{} has some native features to automate the process of creating
a bibliography and textual references.  These are affected with the use
of the \texttt{thebibliography} environment.  This environment will
allow you to define bibliographic references that can be cited in the
main document.  This method is adequate for a document that makes use
of a few sources, but it has some serious shortcomings that make it
cumbersome for use in larger projects.  We will not discuss
\texttt{thebibliography} in detail because it is outclassed by
\BibTeX{}.

Strictly speaking, \BibTeX{} is a separate program from \LaTeX{}, but
it is included with nearly every \LaTeX{} distribution.

To better understand the limitations of this system, consider the
following example:
\begin{verbatim}
\begin{thebibliography}{99}
\bibitem{Widder} Widder, David V.
  \underline{Advanced Calculus}. 2nd ed.
  Mineola, NY: Dover, 1989.
\end{thebibliography}
\end{verbatim}

When placed at the end of a document, this code defines a source with
the key \texttt{Widder}.  To cite this source in the body of the
document, invoke the \texttt{cite} command.
\begin{verbatim}
For more hardcore multivariable calculus, consult
\cite{Widder}.
\end{verbatim}

This will produce the following output:
\begin{quotation}
  For more hardcore multivariable calculus, consult [1].
  
  \ldots{}\textit{more stuff}\ldots{}
  
  [1] Widder, David V. \underline{Advanced Calculus}. 2nd ed. Mineola,
  NY: Dover, 1989.
\end{quotation}
where the reference will print at the end of the document.

\LaTeX{} makes things somewhat easier by handling the indexing.
However, there is not much flexibility in the citation style.  You
must type the citation in the desired style yourself.  In the above
example, the source was type out in typical MLA style.  This can be
frustrating and time consuming if you are not a master in the style
with which you are preparing your document, or if you have to change
the style at some later time.  This inflexibility can prevent you from
using a style correctly.  According to the MLA, \textit{Advanced
  Calculus} should be cited parenthetically as (Widder), rather than
with the index [1].  There is also no easy way to reuse the same
sources in other documents, other than using the traditional cut and
paste.  \BibTeX{} was created to address all of these issues.

\BibTeX{} is not a package; it is an auxiliary program with its own
compiler.  It is not strictly for use with \LaTeX{}; it can be used
with other flavors of \TeX{} as well.  Most \LaTeX{} distributions are
preconfigured to use \BibTeX{}.

The design philosophy of \BibTeX{} is to divorce the bibliographic
information and style from the main document.  The bibliographic
information is stored separately in a \texttt{.bib} (ASCII text)
document and the style is defined by a style package.  An author can
create a large list of commonly-referenced works to be use din many
different documents without repeatedly entering the data.
Furthermore, style packages can be used to automate and customize the
creation and appearance of textual references and the bibliography.  A
good \BibTeX{} installation will include a large number of academic,
legal, and professional styles.  Many journals and publishers provide
their own style packages.  Some individuals have even provided large
\texttt{.bib} documents of works commonly cited in a specific academic
field.  The use of \BibTeX{} may unnecessarily complicate small
documents, but it is an invaluable tool for composing large documents
or many documents that reference the same sources.

\section{The Bibliographic Database}
\label{sec:bibl-datab}

As described above, \BibTeX{} expects the list of references to be
stored in a separate ASCII text document.  To do this, use your text
editor of choice to create a document with a \texttt{.bib} file
extension.  This is where you will type your data.  This example will
use \texttt{sources.bib}.  The format for an entry is relatively
simple.  Figure~\ref{fig:sample-bibtex} provides a sample.

\begin{figure}
  \centering
\begin{boxedverbatim}
@book{Szekeres:2004,
   author = "Peter Szekeres",
   title = {A course in modern mathematical %
     physics: groups, {H}ilbert space and %
     differential geometry},
   publisher = "Cambridge University Press",
   address = Cambridge,
   year = 2004,
}
\end{boxedverbatim}
  \caption{A sample \BibTeX{} entry}
  \label{fig:sample-bibtex}
\end{figure}

Notice that the entry begins with a declaration of the type of
document being referenced (in this case, a book).  This is significant
because books, papers, theses, etc., may be formatted differently,
according to the style package.  Table~\ref{tab:entry-types} is a list
of some of the more common entry types.  For specifics, consult the
documentation of the style package you are using.

\begin{table}
  \centering
  \caption{\BibTeX{} entry types}
  \label{tab:entry-types}
  \resizebox{\textwidth}{!}{\begin{tabular}{@{}ll@{}}
    \toprule
    Type & Description \\
    \midrule
    \texttt{@article} & An article published in a periodical. \\
    \texttt{@book} & A book with a defined author and publisher. \\
    \texttt{@conference} & An article printed in the proceedings of a
    conference. \\
    \texttt{@inproceedings} & Same as \texttt{@conference}. \\
    \texttt{@manual} & A technical document or reference manual. \\
    \texttt{@mastersthesis} & A thesis for a master's degree. \\
    \texttt{@misc} & For works that are hard to categorize (i.e., web
    sites). \\
    \texttt{@phdthesis} & Like \texttt{@mastersthesis}, but for
    doctoral theses. \\
    \bottomrule
  \end{tabular}}
\end{table}

The entire entry is enclosed in braces.  Following the declaration of
the entry type, it is necessary to assign the entry a key name.  This
is the name by which the source will be referenced in the main
document.  In this case, the entry is named according to the author
and the date of publication, \texttt{Szekeres:2004}.

Commas are used to separate the data fields.  Neglecting a comma at
the end of a field is the most common source of errors with \BibTeX{}.
Each field is assigned a value with the = operator.  You must enclose
strings with multiple words in quotation marks or braces, otherwise
the \BibTeX{} compiler will ignore the whitespace and become confused.

Table~\ref{tab:data-fields} is a partial list of the fields that
\BibTeX{} understands.  Not all fields may be displayed by a
bibliographic entry; this depends on the style being used.

\begin{table}
  \centering
  \caption{\BibTeX{} data fields}
  \label{tab:data-fields}
  \resizebox{\textwidth}{!}{\begin{tabular}{@{}ll@{}}
    \toprule
    Field & Description \\
    \midrule
    \texttt{address} & The address (usually just the city and state)
    of the publisher. \\
    \texttt{author} & The author(s); special considerations to be
    discussed shortly. \\
    \texttt{booktitle} & The title of the book, when citing an article
    printed in a book. \\
    \texttt{edition} & The edition of the book; write it out in
    full. \\
    \texttt{editor} & The editor(s); refer to the discussion of the
    author name. \\
    \texttt{journal} & The name of the periodical in which an article
    is published. \\
    \texttt{month} & The month of publication for a periodical; can be
    abbreviated. \\
    \texttt{note} & For information that does not fit into any other
    category. \\
    \texttt{publisher} & The name of the publisher.  What else? \\
    \texttt{title} & The title of the article or book. \\
    \texttt{url} & The uniform resource locator of a document accessed
    online. \\
    \texttt{volume} & The volume number of a journal or multiple
    volume book. \\
    \texttt{year} & The year of publication. \\
    \bottomrule
  \end{tabular}}
\end{table}

Notice the use of capitalization in Figure~\ref{fig:sample-bibtex}.
Some citation styles have different conventions for capitalization in
the document title.  For example, MLA style capitalizes every word
except prepositions, whereas only the first word of a title and proper
nouns are capitalized according to the Chicago style.  It is best to
let the style package handle the capitalization.  Surrounding a word
or letter in braces will force capitalization, as in
\texttt{\{H\}ilbert}.  This is best reserved for proper nouns.  \TeX{}
commands for special characters can be used, such as \verb|\"u| for
\"u, but should be enclosed in braces as well.

Author names can either be entered as \textit{"Given name Surname"} or
\textit{"Surname, Given name"} (i.e., \textit{"Grover Cleveland"} or
\textit{"Cleveland, Grover"}).  \BibTeX{} will format the author's
name appropriately.  If there is any ambiguity between the given name
and the surname, use braces (for example, use \textit{"Zack \{de la
  Rocha\}"} instead of \textit{"Zack de la Rocha"}).  Generally, the
use of braces forces \BibTeX{} to treat the enclosed quantity as a
single entity.

To reference a work by multiple authors, simply insert \texttt{and}
between each name.  The style package will handle the rest.  The list
of authors may automatically be shortened and terminated with
\textit{et al}, \textit{and others}, etc., if it becomes too long.

In general, an entry follows the following syntax:
\begin{verbatim}
@doctype{<key>,
    author = {...},
    title = {...},
    ...,
}
\end{verbatim}

A number of entries constitutes a bibliography database (hencefore,
``source file'').  If you notice a commonly recurring stirng in your
sources, such as multiple references to the same journal, you can
define a macro to use as an abbreviation.  For example, to abbreviate
\texttt{The American Journal of Physics} as \texttt{AJP}, place the
following declaration at the beginning of the source file:
\begin{verbatim}
@string{AJP = {The American Journal of Physics}}
\end{verbatim}

Then to reference the journal in a bibliographic entry, define the
journal field with \texttt{journal = ajp}.  The abbreviation is not
case-sensitive.  The general format for a string declaration is
\begin{verbatim}
@string{abbr_name = {...} }
\end{verbatim}

\section{Generating the Bibliography}
\label{sec:gener-bibl}

You must include a style package in order to produce a bibliography in
your document.  There are plenty of styles available online and you
will most likely find one that fits your needs.  The three most common
styles in the humanities are those defined by the Modern Language
Association (MLA style), the American Psychology Association (APA
style), and the University of Chicago's Manual of Style (Chicago
style).  Only high school teachers (and some literature professors)
care about MLA style, but implementations of the others hsould be in
any respectable \LaTeX{} distribution.

To generate a bibliography with \BibTeX{}, you may need to include a
macro package.  A style may require macros that are defined in a
specific package.  The \texttt{natbib} package seems to work with most
styles.  Some styles come bundled with their own macro package:
\texttt{natbib} includes some styles, and so does \texttt{jurabib}.
Some common styles are listed in Table~\ref{tab:style-packages}.  In
the document body, define the style you want by passing it to the
\verb|\bibliographystyle| command.  At the point in the body where you
want the bibliography to appear (usually at the end), pass the name of
your source file to the \verb|\bibliography| command.

\begin{table}
  \centering
  \caption{Common \BibTeX{} style packages}
  \label{tab:style-packages}
  \begin{tabular}{@{}ll@{}}
    \toprule
    Style Package & Description \\
    \midrule
    \texttt{apa} & An implementation of APA style. \\
    \texttt{apacite} & Another APA style package. \\
    \texttt{apalike} & It's like APA, but not quite. \\
    \texttt{chicago} & Do it Chicago style. \\
    \texttt{chicagoa} & \ldots{}another Chicago style. \\
    \texttt{harvard} & Harvard style?  Does anyone use this? \\
    \texttt{jurabib} & A few useful styles, as well as footnote
    support. \\
    \texttt{natbib} & Several styles common in the natural
    sciences. \\
    \texttt{oxford} & Like Harvard style, but from Oxford. \\
    \bottomrule
  \end{tabular}  
\end{table}

As a side effect of \BibTeX{} actually being a separate program, you
must run both programs on your source file.  First, use \LaTeX{} to
compile the document and ignore all of the compiler warnings.  Then
run the \BibTeX{} compiler.  In a properly configured \LaTeX{}
environment, this is done by inputting
\begin{verbatim}
bibtex documentname.tex
\end{verbatim}
at the command line.  Some editors have support for \BibTeX{} in the
interface.  Assuming that \BibTeX{} executes properly, you can
recompile the document with \LaTeX{} and the citations and
bibliography should appear.  If you notice errors, such as lots of
question marks in your document, or if the \LaTeX{} compiler keeps
complaining about missing references, run \LaTeX{} again.  It can take
several compiles for all the references to stabilize.

If you make changes to your source file, you will need to recompile
with \BibTeX{} and again with \LaTeX{} to update all of the
references.

As an example, let us say there was a paper that used a variant of the
Oxford style that required the \texttt{jurabib} package (it actually
comes with the package).  The style itself is called \texttt{jox}.
The name of the bibliography database was \texttt{sources.bib}.
Included in the document preamble was
\begin{verbatim}
\usepackage{jurabib}
\end{verbatim}
and just before the end of the main body was
\begin{verbatim}
\bibliographystyle{jox}
\bibliography{sources}
\end{verbatim}
Of course, the \texttt{jurabib} package needed to be installed
correctly for this to work (see Chapter~\ref{cha:packages}).  Also,
the file \texttt{sources.bib} was located in the same directory as the
\LaTeX{} document.  If it were somewhere else, its exact location must
be specified.  \BibTeX{} expects the source file to have a
\texttt{.bib} extension; the complete filename would have been needed
if the extension was something else.

The \texttt{babelbib} package will let you specify a language for your
bibliography.  Also, with a package like \texttt{bibunits} or
\texttt{multibib}, you can include more than one bibliography in the
same document, say, one for each chapter.  The moderately useful
\texttt{splitbib} package will allow you to split up a bibliography
and organize it by category.  Check out their documentation to find
out how to do this.

\section{Textual References}
\label{sec:textual-references}

\BibTeX{} will only include those references in the source file that
are cited in your document.  If you had created the source file
sampled above and invoked \BibTeX{} in some random document, nothing
would have happened (or, nothing \emph{should} have happened).

The citation method will depend somewhat on your choice of style.  It
is generally accomplished by passing the key of the document you are
referencing to the \verb|\cite| macro.  To cite our previous example:
\begin{verbatim}
The theory of general relativity postulates that
spacetime is a four-dimensional Minkowskian
manifold \cite{Szekeres:2004}.
\end{verbatim}
This will produce an inline citation, automatically formatted by the
style.  Using the \texttt{chicago} style, this would print as
(Szekeres, 2004).  Most styles will allow you to pass more information
to \verb|\cite|, such as the page number, i.e.,
\verb|\cite[536]{Szekeres:2004}| to cite page 536.  This input gives
(Szekeres, 2004, 536) in \texttt{chicago}.

You can force the inclusion of a document that you do not explicitly
cite with \verb|\nocite{<key>}|.  This essentially creates an
invisible citation.  For styles that do not default to parenthetical
citations, you can usually use \verb|\citep{<key>}| to make one.  The
\texttt{footbib} package introduces the \verb|\footcite{<key>}| macro
that will place the full citation in the page footer, although it only
works with some styles.  This is also possible with the
\texttt{jurabib} package.  Finally, if you just want to list every
single entry in your source file in the bibliography, use
\verb|\nocite{*}|.

%%% Local Variables: 
%%% mode: latex
%%% TeX-master: "reader"
%%% End: 